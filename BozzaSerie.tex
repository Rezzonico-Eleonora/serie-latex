\documentclass[a4paper,10pt]{article}

\usepackage{amsmath,amssymb}
\usepackage{epsfig}
\usepackage{cancel}
\usepackage{color}
\usepackage{wasysym}
\usepackage{pifont}

\newcommand{\hili}[1]{\colorbox{yellow}{#1}}

\textwidth16cm
\textheight25cm
\topmargin0mm
\headheight0mm
\headsep0mm
\oddsidemargin0mm
\evensidemargin0mm
\parindent0mm

\newcommand{\df}{\displaystyle\frac}
\def\arc#1{\overset{\;\rotatebox{90}{)}}{#1}}


\begin{document}
\pagestyle{empty}


\begin{center}
\LARGE{\bf \underline{SERIE 1414}}\\[1mm]
\end{center}
Mettere nel titolo la serie 1414/soluzioni serie 1414 corrispondente!\\\\

Vanno trascritte sia le serie che le soluzioni. Avete capito?!? 
\`E tutto da ricopiare!\\
Il titolo di ogni esercizio va scritto all'interno di un riquadro come segue\footnote{Pure per le soluzioni utilizzare la stessa impaginazione!}:\\\\

\fbox{\bf Esercizio 1}\\\\
Trascrivere consegna esercizio 1.\\
$x^2$

\vspace*{1cm}
\fbox{\bf Esercizio 2}\\\\
Trascrivere consegna esercizio 2. Ad esempio: qual \`e la formula risolutiva?\\\\
$x_1,2=\dfrac{-88b\pm\sqrt{-\Delta}}{27a}$ con $\Delta=b^{2000}-404ac\cdot\pi^{17}$\\\\
NO!!!!!!!!!!!!!!!!!!!!!!!!!!!!!!!!!!!!!!!!!!!!!!!!!!!!!!!!!!!!!!!!!!!!!!!! \`E:\\

$x_{1,2}=\dfrac{-b\pm\sqrt{\Delta}}{2a}$ con $\Delta:=b^2-4ac$\\\\


Pure per le soluzioni si procede allo stesso modo dove:
\begin{itemize}
\item[\textbf{a)}] La soluzione della serie \`e da scrivere con tutti i passaggi!
\item[\textbf{b)}] Eventuali trucchetti o formule importanti sono da evidenziare cos\`i:\\
\boxed{\hili{$E=mc^2$}}
\item[\textbf{c)}] Eventuali immagini possono essere inserite come segue:
\begin{center}
\epsfig{file=Tex1,width=8cm}
\end{center}
Ricordatevi di salvare tali immagini nella stessa cartella del file .tex altrimenti non compila! Capito?!?
\item[\textbf{d)}] Eventuali strategie risolutive sono da enfatizzare come segue:\\\\
\begin{minipage}{3cm}
\begin{center}
\epsfig{file=attenzione ,width=2cm }
\end{center}
\end{minipage}
\hfill
\begin{minipage}{12.5cm}
Per le funzioni fratte :\\\\ \textbf{il denominatore non pu\`o essere uguale a $0$\,!}
\end{minipage} 
\item[\textbf{e)}] Adesso per\`o $\cdots$ e $\ldots$ basta veramente chiacchierare: \textit{``state disturbando il regolare flusso della lezione!''} :-(
\end{itemize}



\begin{center}
\emph{BUON LAVORO A TUTTI!!!}
\end{center}

ciao come va?


\end{document} 
\documentclass[a4paper,10pt]{article}

% packages

\usepackage{amsmath,amssymb}
\usepackage{epsfig}
\usepackage{cancel}
\usepackage{color}
\usepackage{wasysym}
\usepackage{pifont}

% added
\usepackage{mathtools}
\usepackage{enumitem}
% /added

% /packages


% layout

\textwidth16cm
\textheight25cm
\topmargin0mm
\headheight0mm
\headsep0mm
\oddsidemargin0mm
\evensidemargin0mm
\parindent0mm

% /layout

% shortcuts

\newcommand{\df}{\displaystyle\frac}

% /shortcuts

\begin{document}
\pagestyle{plain}

% TITLE

\begin{center}
    \LARGE{\bf SERIE 7}

    \includegraphics[scale=0.5]{./img.jpg}
\end{center}

% /TITLE

%---------%
\smallskip
%---------%


% EX1

\fbox{\bf Esercizio 1} \\\\
Calcola il determinante delle seguenti matrici il più velocemente possibile:

\begin{table}[h!]
\centering
\begin{tabular}{ c c c }
    % 1-1
    $ A= 
        \begin{pmatrix} 
            1 + \sqrt{2} & 2 - \sqrt{3} \\
            2 + \sqrt{3} & 1 - \sqrt{2} \\
        \end{pmatrix} 
    $ 

    &

    % 1-2
    $ B= 
        \begin{pmatrix} 
            1 & \log_{b}{a} \\
            \log_{a}{b} & 1 \\
        \end{pmatrix} 
    $

    &

    % 1-3
    $ C= 
        \begin{pmatrix} 
            \sin{t} & \cos{t} \\
            -\cos{t} & \sin{t} \\
        \end{pmatrix}
    $

    \\ [6ex] % row end

    % 2-1
    $ D= 
        \begin{pmatrix} 
            a & -a & a \\
            a & a & -a \\
            a & -a & -a 
        \end{pmatrix} 
    $ 

    &

    % 2-2
    $ E= 
        \begin{pmatrix} 
            1 & a & 2a + 2 \\
            2 & a & 2a + 4 \\
            3 & a & 2a + 6 
        \end{pmatrix} 
    $

    &

    % 2-3
    $ F= 
        \begin{pmatrix} 
            1 & 7 & -3 \\
            0 & 0 & 0 \\
            2 & -6 & 1
        \end{pmatrix} 
    $

    \\ [6ex] % row end

    % 3-1
    $ G= 
        \begin{pmatrix} 
            2 & 4 & 1 \\
            8 & 7 & 4 \\
            -6 & 4 & -3 
        \end{pmatrix} 
    $  

    &

    % 3-2
    $ H=  
        \begin{pmatrix} 
            1 & -8 & 3 \\
            2 & 2 & 2 \\
            2 & -16 & 6 
        \end{pmatrix} 
    $

    &

    % 3-3
    $ I= 
        \begin{pmatrix} 
            sin{x}cos{x} & vcos{x}cos{y} & -vsin{x}sin{y} \\
            sin{x}sin{y} & vcos{x}sin{y} & vsin{x}cos{y} \\
            cos{x} & -vsin{x} & 0
        \end{pmatrix} 
    $

    \\ [6ex] % row end

    % 4-1
    $ L=  
        \begin{pmatrix} 
                2 & -1 & 0 & 1 \\
                0 & 0 & 2 & 5 \\
                0 & 4 & 1 & 0 \\
                4 & 5 & -1 & 1  
        \end{pmatrix} 
    $        
    &

    % 4-2
    $ M= 
        \begin{pmatrix} 
            1 & 2 & 3 & 4 & 5 \\
            6 & 7 & 8 & 9 & 10 \\
            11 & 12 & 13 & 14 & 15 \\
            a & b & c & d & e \\
            f & g & h & i & j
        \end{pmatrix} 
    $

    &

    % 4-3
    $ N= 
        \begin{pmatrix} 
            0 & 0 & 0 & 0 & 0 & 6 \\
            0 & 0 & 0 & 0 & 5 & 5 \\
            0 & 0 & 0 & 4 & 12 & -4 \\
            0 & 0 & 3 & 0 & 6 & 3 \\
            0 & 2 & 2 & 0 & 2 & 2 \\
            1 & 8 & 5 & 1 & 20 & 1 
        \end{pmatrix} 
    $

    \\ [6ex] % row end

    \end{tabular}
\end{table}

% /EX1 


%--------%
\bigskip
%--------%


% EX2

\fbox{\bf Esercizio 2} \\\\
    Sia $A$ una matrice di tipo $7 \times 7$ con $\det(A)=3$
    \begin{enumerate}
        \item Calcola $\det(A^{T})$ e $\det(2A)$
        \item Determina $\det(B)$, in modo che valga $\det(AB) = -27$
    \end{enumerate}

% /EX2


%-------%
\bigskip
%-------%


% EX3

\fbox{\bf Esercizio 3} \\\\
    Dimostra la regola di Sarrus utilizzando il Teorema di Laplace


% /EX3

%==========%
\pagebreak
%==========%

% EX4

\fbox{\bf Esercizio 4} \\\\
    Dimostra che il seguente determinante

        $$ 
        \begin{vmatrix}
            a_{0} & a_{1} & a_{2} & a_{3} & a_{4} \\
            -1 & x & 0 & 0 & 0 \\
            0 & -1 & x & 0 & 0 \\
            0 & 0 & -1 & x & 0 \\
            0 & 0 & 0 & -1 & x 
        \end{vmatrix}
        $$
    è uguale a $\displaystyle\sum\limits_{i=0}^{4}{a_{1} \cdot x^{4-i}}$.

    %
    %   Due possibilità
    %   1. $\sum_{i=0}^{4}{a_{1} \cdot x^{4-i}}$
    %       --> all'interno del testo con i pedici e gli apici della sommatoria
    %           in posizione "orizzontale"
    %   2. $$\sum_{i=0}^{4}{a_{1} \cdot x^{4-i}}$$
    %       --> si crea un blocco a sé in mezzo al foglio, ma i pedici e gli apici
    %           sono mostrati superiormente e inferiormente
    %
    %   Soluzione
    %   $\sum\limits_{i=0}^{4}{a_{1} \cdot x^{4-i}}$.
    %   
    %   Considerare la possibilità di modificare: 
    %   $\displaystyle\sum\limits_{i=0}^{4}{a_{1} \cdot x^{4-i}}$
    %       --> migliero visiblità e leggibilità
    %   MIGLIORE SOLUZIONE!

% /EX4


%--------%
\bigskip
%--------%


% EX 5
\fbox{\bf Esercizio 5} \\\\
    Il determinante $\Delta_{n}$ di Vandermonde per le incognite $x_{1}, \dots, x_{n}$ è definito come segue:

        $$
        \Delta_{n} \coloneqq \det{
        \begin{pmatrix}
            1 & x_{1} & x_{1}^{2} & \cdots & x_{1}^{n-1} \\
            \vdots & \vdots & \vdots & & \vdots \\
            1 & x_{n} & x_{n}^{2} & \cdots & x_{n}^{n-1}
        \end{pmatrix}
        }
        $$

    Dalla definizione sopra è ovvio che vale $\Delta_{1}=1, \Delta_{2}=x_{2}-x_{1}$.
    \begin{enumerate}
        \item Calcola $\Delta_{3}$ e $\Delta_{4}$
        \item Dimostra per induzione completa che $\Delta_{n}=\displaystyle\prod_{ 1 \leq i < j \leq n }(x_{j} - x_{i})$
    \end{enumerate}



    %
    %   RICONTROLLARE FORMATTAZIONE E LAYOUT
    %

    \setlength{\parindent}{5ex}

    \fbox{Aiuto} 
    \begin{itemize}[leftmargin=10ex]        
        \item
            Per la base d'induzione utilizza il fatto che il "prodotto vuoto" 
            $\Delta_{1}=\displaystyle\prod_{ 1 \leq i < j \leq 1 }(x_{j} - x_{i})$
            è uguale a $1$ per definizione
        \item per il passo d'induzione utilizza opportune operazioni elementari sulle colonne.
    \end{itemize}

    \setlength{\parindent}{0ex}

% /EX5


%--------%
\bigskip
%--------%


% EX 6
\fbox{\bf Esercizio 6} \\\\
    Considera i due punti \underline{distinti} 
    $v\coloneqq(v_{1},v_{2})$
    e
    $w\coloneqq(w_{1},w_{3})$
    di $\mathbb{R}^{2}$
    e la retta $L \subset \mathbb{R}^{2}$ passante per $v$ e $w$.
    \\
    Dimostra che allora vale:

    %
    % CONTROLLARE FORMATTAZIONE LINEA VERTICALE
    %

    $$L= \{ (x_{1}, x_{2}) \in \mathbb{R}^{2} \vert det{
        \begin{pmatrix}
            1 & v_{1} & v_{2} \\
            1 & w_{1} & w_{2} \\
            1 & x_{1} & x_{2}
        \end{pmatrix}
    } = 0 \}$$

    %
    % CONTROLLARE FORMATTAZIONE E LAYOUT
    %

    \underline{AIUTO:} 
    Mostra che l'insieme $L$ definito sopra è equivalente alla forma parametrica
    di tale retta in $\mathbb{R}^{2}$

    % /EX6

\end{document}
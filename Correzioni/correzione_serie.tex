\documentclass[a4paper,10pt]{article}

% packages

\usepackage{amsmath,amssymb}
\usepackage{epsfig}
\usepackage{cancel}
\usepackage{color}
\usepackage{wasysym}
\usepackage{pifont}

% added
\usepackage{mathtools}
\usepackage{enumitem}
%\usepackage{ulem}
\usepackage{cancel}
% /added

% /packages


% layout

\textwidth16cm
\textheight25cm
\topmargin0mm
\headheight0mm
\headsep0mm
\oddsidemargin0mm
\evensidemargin0mm
\parindent0mm

% /layout

% shortcuts

\def\doubleu#1{\underline{\underline{#1}}}

\newcommand\df{\displaystyle\frac}

\def\canc#1{\displaystyle\cancel{#1}}

\def\over#1#2{
        \mathrel{
            \overset{
                \makebox[0pt]{
                    \mbox{\normalfont\tiny\sffamily #2}
                }
            }{#1}
        }   
    }

\def\under#1#2{\mathrel{\underset{\makebox[0pt]{\mbox{\normalfont\tiny\sffamily #2}}}{#1}}}

\def\rom#1{\uppercase\expandafter{\romannumeral #1\relax}}

% /shortcuts

\begin{document}
\pagestyle{plain}

% TITLE

\begin{center}
    \LARGE{\bf SERIE 7}\\
    \large{CORREZIONI}
\end{center}

% /TITLE

%---------%
\smallskip
%---------%


% EX1

\fbox{\bf Esercizio 1} \\

%\begin{table}[h!]
%\centering
%\begin{tabular}{ l l l}
    % 1-1
    $\det(A)=\doubleu{-2}$\\ 

    %&

    % 1-2
    $ 
    \det(B)= 1 - \log_{a}(b) \cdot \log_{b}(a) =
    1 - \df{\cancel{\log{b}}}{\cancel{\log{a}}} \cdot 
    \df{\cancel{\log{a}}}{\cancel{\log{b}}} = \doubleu{0}
    $\\

    %\\ [3ex] % row end

    % 1-3
    $ 
    \det(C) = \sin(t)^{2} + \cos(t)^{2} = \doubleu{1}
    $\\

    %&

    % 2-1
    $ 
    \det(D) \over{=}{Sarrus} \doubleu{-4a^{3}}
    $ \\


    %\\ [3ex] % row end


    % 2-2
    $ 
    \det(E) = \doubleu{0}
    $
      siccome 
    $\rom3^{c} = 2\rom2^{c} + 2\rom1^{c}$
    ossia colonne linearmente dipendenti\\

    %\\ [3ex] % row end

    % 2-3
    $\det(F) = \doubleu{0}$
    seconda riga pari a $0$\\

    %\\ [3ex] % row end

    % 3-1
    $ \det(G) = 
        \begin{pmatrix} 
            2 & 4 & 1 \\
            8 & 7 & 4 \\
            -6 & 4 & -3 
        \end{pmatrix} 
    $

    %&

    % 3-2
    $ H=  
        \begin{pmatrix} 
            1 & -8 & 3 \\
            2 & 2 & 2 \\
            2 & -16 & 6 
        \end{pmatrix} 
    $

    %&

    % 3-3
    $ I= 
        \begin{pmatrix} 
            sin{x}cos{x} & vcos{x}cos{y} & -vsin{x}sin{y} \\
            sin{x}sin{y} & vcos{x}sin{y} & vsin{x}cos{y} \\
            cos{x} & -vsin{x} & 0
        \end{pmatrix} 
    $

    %\\ [6ex] % row end

    % 4-1
    $ L=  
        \begin{pmatrix} 
                2 & -1 & 0 & 1 \\
                0 & 0 & 2 & 5 \\
                0 & 4 & 1 & 0 \\
                4 & 5 & -1 & 1  
        \end{pmatrix} 
    $

    %&

    % 4-2
    $ M= 
        \begin{pmatrix} 
            1 & 2 & 3 & 4 & 5 \\
            6 & 7 & 8 & 9 & 10 \\
            11 & 12 & 13 & 14 & 15 \\
            a & b & c & d & e \\
            f & g & h & i & j
        \end{pmatrix} 
    $

    %&

    % 4-3
    $ N= 
        \begin{pmatrix} 
            0 & 0 & 0 & 0 & 0 & 6 \\
            0 & 0 & 0 & 0 & 5 & 5 \\
            0 & 0 & 0 & 4 & 12 & -4 \\
            0 & 0 & 3 & 0 & 6 & 3 \\
            0 & 2 & 2 & 0 & 2 & 2 \\
            1 & 8 & 5 & 1 & 20 & 1 
        \end{pmatrix} 
    $

    %\\ [6ex] % row end

    %\end{tabular}
%\end{table}

% /EX1 


%--------%
\bigskip
%--------%


% EX2

\fbox{\bf Esercizio 2} \\\\
    Sia $A$ una matrice di tipo $7 \times 7$ con $\det(A)=3$
    \begin{enumerate}
        \item Calcola $\det(A^{T})$ e $\det(2A)$
        \item Determina $\det(B)$, in modo che valga $\det(AB) = -27$
    \end{enumerate}

% /EX2


%-------%
\bigskip
%-------%


% EX3

\fbox{\bf Esercizio 3} \\\\
   \underline{Regola di Sarrus:}\\


${\begin{vmatrix} 
a_{11} & a_{12} & a_{13} \\
a_{21} & a_{22} & a_{23} \\
a_{31} & a_{32} & a_{33} 
\end{vmatrix} = a_{11}a_{22}a_{33} + a_{12}a_{23}a_{31} + a_{13}a_{21}a_{32} - a_{31}a_{22}a_{13} - a_{32}a_{23}a_{11}-a_{33}a_{21}a_{12}}$
\\
\\
\underline{Dimostrazione:}
\\

$\text{Utilizziamo il Teorema di Laplace sviluppando secondo la prima riga:}$
\\


${{\begin{vmatrix}
a_{11} & a_{12} & a_{13} \\
a_{21} & a_{22} & a_{23} \\
a_{31} & a_{32} & a_{33}
\end{vmatrix}}= (-1)^{1+1}\cdot a_{11} \cdot 
\begin{vmatrix}
a_{22} & a_{23}\\
a_{32} & a_{33}
\end{vmatrix}
+ (-1)^{1+2} \cdot a_{12} \cdot 
\begin{vmatrix}
a_{21} & a_{23} \\
a_{31} & a_ {33}
\end{vmatrix}
+ (-1)^{1+3} \cdot a_{13} \cdot 
\begin{vmatrix}
a_{21} & a_{22} \\
a_{31} & a_{32}
\end{vmatrix}=}$ 
\\
\\
$a_{11}a_{22}a_{33} - a_{11}a_{32}a_{23} - a_{12}a_{21}a_{33} + a_{12}a_{31}a_{23} + a_{13}a_{21}a_{32} - a_{13}a_{31}a_{22}$
\\
\\
 $\text{tutti i termini, seppur in ordine differente, corrispondono.}$


% /EX3

%==========%
\pagebreak
%==========%

% EX4

\fbox{\bf Esercizio 4} \\\\
   $\text{Utilizzando il Teorema di Laplace sviluppando secondo la prima colonna.}$
\\
\\
 $\begin{vmatrix}
 a_0 & a_1 & a_2 & a_3 & a_4 \\
 -1 & x & 0 & 0 & 0 \\
 0 & -1 & x & 0 & 0 \\
 0 & 0 & -1 & x & 0 \\
 0 & 0 & 0 & -1 & x
 \end{vmatrix}$
 $ = a_0 \cdot x^4 + 1 \cdot 
 \begin{vmatrix}
 a_1 & a_2 & a_3 & a_4\\ 
 -1 & x & 0 & 0 \\
 0 & -1 & x & 0 \\
 0 & 0 & -1 & x
 \end{vmatrix}$
 \\
 \\
  $\text{e sempre sviluppando la prima colonna ci accorgiamp che il tutto si ripte sempre in modo analogo:}$
 \\
 \\
 ${= a_0 \cdot x^4 + a_1 \cdot x^3 + 1 \cdot 
  \begin{vmatrix}
 a_2 & a_3 & a_4\\
  -1 & x & 0\\
  0 & -1 & x
  \end{vmatrix}
\stackrel{Sarrus}{=} a_0 \cdot x^4 + a_1 \cdot x^3 + a_2 \cdot x^2 + a_4 + a_3 \cdot x =  	\sum_{i=0}^4 a_i \cdot x^{4-i}}$ 

% /EX4


%--------%
\bigskip
%--------%


% EX 5
\fbox{\bf Esercizio 5} \\\\
    Il determinante $\Delta_{n}$ di Vandermonde per le incognite $x_{1}, \dots, x_{n}$ è definito come segue:

        $$
        \Delta_{n} \coloneqq \det{
        \begin{pmatrix}
            1 & x_{1} & x_{1}^{2} & \cdots & x_{1}^{n-1} \\
            \vdots & \vdots & \vdots & & \vdots \\
            1 & x_{n} & x_{n}^{2} & \cdots & x_{n}^{n-1}
        \end{pmatrix}
        }
        $$

    Dalla definizione sopra è ovvio che vale $\Delta_{1}=1, \Delta_{2}=x_{2}-x_{1}$.
    \begin{enumerate}
        \item Calcola $\Delta_{3}$ e $\Delta_{4}$
        \item Dimostra per induzione completa che $\Delta_{n}=\displaystyle\prod_{ 1 \leq i < j \leq n }(x_{j} - x_{i})$
    \end{enumerate}



    %
    %   RICONTROLLARE FORMATTAZIONE E LAYOUT
    %

    \setlength{\parindent}{5ex}

    \fbox{Aiuto} 
    \begin{itemize}[leftmargin=10ex]        
        \item
            Per la base d'induzione utilizza il fatto che il "prodotto vuoto" 
            $\Delta_{1}=\displaystyle\prod_{ 1 \leq i < j \leq 1 }(x_{j} - x_{i})$
            è uguale a $1$ per definizione
        \item per il passo d'induzione utilizza opportune operazioni elementari sulle colonne.
    \end{itemize}

    \setlength{\parindent}{0ex}

% /EX5


%--------%
\bigskip
%--------%


% EX 6
\fbox{\bf Esercizio 6} \\\\
    $\text{Dalla seconda liceo sappiamo che la forma parametrica della retta L passante per i due punti} \\
v=(v_1,v_2) e w=(w_1,w_2) \text{è data da:}$
\\
\\
$L= \binom{x_1}{x_2}= \binom{v_1}{v_2} + \lambda \binom{w_1 - v_1}{w_2 -_2},$
\\
\\
$\text{in altre parole vale che}: x_i - v_i = \lambda \cdot (w_i - v_i) \text{per i}=1,2.$
\\
\\
$\text{Parallelamente calcoliamo il determinante dato:}$
\\
\\
$\begin{vmatrix}
1 & v_1 & v_2 \\
1 & w_1 & w_2 \\
1 & x_1 & x_2 
\end{vmatrix}
\stackrel{II-I}{\stackrel{III-I}{=}} $
$\begin{vmatrix}
1 & v_1 & v_2 \\
0 & w_1-v_1 & w_2 -v_2 \\
0 & x_1 - v_1 & x_2 - v_2
\end{vmatrix}
\stackrel{Laplace}{=} 1\cdot 
\begin{vmatrix} 
w_1 - v_1 & w_2 -v_2 \\
x_1 -v_1 & x_2 -v_2 
\end{vmatrix}.$
\\
\\
\\
$\text{Ora, siccome L è definito quando il determinante}
\begin{vmatrix}
w_1-v_1 & w_2-v_2 \\
x_1-v_1 & x_2-v_2
\end{vmatrix}
\\
\text{è uguale a zero, 
 questo equivale a dire che:}$
 \\
 \\
$(w_1-v_1) \cdot (x_2 -v_2) - (w_2 -v_2) \cdot (x_1 -v_1) = 0 
\\
\\
\leftrightarrow(w_1 -v_1) \cdot (x_2-v_2) = (w_2 -v_2) \cdot (x_1 -v_1)$
\\
\\
$ \text{cioè} \exists \lambda \in \mid \lambda \cdot (w_i -v_i) = (x_i - v_i) \text{per i}=1,2.$

    % /EX6

\end{document}

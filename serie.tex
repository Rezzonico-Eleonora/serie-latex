\documentclass[a4paper,10pt]{article}

% doc-info
    \title{Serie 7}
    \author{Name}
% packages

\usepackage{amsmath,amssymb}
\usepackage{epsfig}
\usepackage{cancel}
\usepackage{color}
\usepackage{wasysym}
\usepackage{pifont}

% added

\usepackage{array}

% /added

% /packages


% layout

\textwidth16cm
\textheight25cm
\topmargin0mm
\headheight0mm
\headsep0mm
\oddsidemargin0mm
\evensidemargin0mm
\parindent0mm


% shortcuts

\newcommand{\df}{\frac}

% /shortcuts

\begin{document}
\pagestyle{plain}


\begin{center}
    \LARGE{\bf SERIE 7}

    \includegraphics[scale=0.5]{./img.jpg}
\end{center}

\smallskip


\fbox{\bf Esercizio 1} \\\\
Calcola il determinante delle seguenti matrici il più velocemente possibile:

\begin{table}[h!]
\centering
\begin{tabular}{ c c c }
    % 1-1
    $ A= 
        \left( 
            \begin{matrix} 
                1 + \sqrt{2} & 2 - \sqrt{3} \\
                2 + \sqrt{3} & 1 - \sqrt{2} \\
            \end{matrix} 
        \right)
    $ 

    &

    % 1-2
    $ B= 
        \left( 
            \begin{matrix} 
                1 & \log_{b}{a} \\
                \log_{a}{b} & 1 \\
            \end{matrix} 
        \right)
    $

    &

    % 1-3
    $ C= 
        \left( 
            \begin{matrix} 
                \sin{t} & \cos{t} \\
                -\cos{t} & \sin{t} \\
            \end{matrix} 
        \right)
    $

    \\ [6ex] % row end

    % 2-1
    $ D= 
        \left( 
            \begin{matrix} 
                a & -a & a \\
                a & a & -a \\
                a & -a & -a 
            \end{matrix} 
        \right)
    $ 

    &

    % 2-2
    $ E= 
        \left( 
            \begin{matrix} 
                1 & a & 2a + 2 \\
                2 & a & 2a + 4 \\
                3 & a & 2a + 6 
            \end{matrix} 
        \right)
    $

    &

    % 2-3
    $ F= 
        \left( 
            \begin{matrix} 
                1 & 7 & -3 \\
                0 & 0 & 0 \\
                2 & -6 & 1
            \end{matrix} 
        \right)
    $

    \\ [6ex] % row end

    % 3-1
    $ G= 
        \left( 
            \begin{matrix} 
                2 & 4 & 1 \\
                8 & 7 & 4 \\
                -6 & 4 & -3 
            \end{matrix} 
        \right)
    $  

    &

    % 3-2
    $ H= 
        \left( 
            \begin{matrix} 
                1 & -8 & 3 \\
                2 & 2 & 2 \\
                2 & -16 & 6 
            \end{matrix} 
        \right)
    $

    &

    % 3-3
    $ I= 
        \left( 
            \begin{matrix} 
                sin{x}cos{x} & vcos{x}cos{y} & -vsin{x}sin{y} \\
                sin{x}sin{y} & vcos{x}sin{y} & vsin{x}cos{y} \\
                cos{x} & -vsin{x} & 0
            \end{matrix} 
        \right)
    $

    \\ [6ex] % row end

    % 4-1
    $ L= 
        \left( 
            \begin{matrix} 
                2 & -1 & 0 & 1 \\
                0 & 0 & 2 & 5 \\
                0 & 4 & 1 & 0 \\
                4 & 5 & -1 & 1  
            \end{matrix} 
        \right)
    $        
    &

    % 4-2
    $ M= 
        \left( 
            \begin{matrix} 
                1 & 2 & 3 & 4 & 5 \\
                6 & 7 & 8 & 9 & 10 \\
                11 & 12 & 13 & 14 & 15 \\
                a & b & c & d & e \\
                f & g & h & i & j
            \end{matrix} 
        \right)
    $

    &

    % 4-3
    $ N= 
        \left( 
            \begin{matrix} 
                0 & 0 & 0 & 0 & 0 & 6 \\
                0 & 0 & 0 & 0 & 5 & 5 \\
                0 & 0 & 0 & 4 & 12 & -4 \\
                0 & 0 & 3 & 0 & 6 & 3 \\
                0 & 2 & 2 & 0 & 2 & 2 \\
                1 & 8 & 5 & 1 & 20 & 1 
            \end{matrix} 
        \right)
    $

    \\ [6ex] % row end

    \end{tabular}
\end{table}



% INSERIRE TITOLO E IMMAGINE


$ \df {1} {2}$

\end{document}


